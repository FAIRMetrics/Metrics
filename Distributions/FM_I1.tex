\documentclass[english]{article}
\usepackage[utf8]{inputenc}
\usepackage{babel}
\usepackage{color}
%\usepackage{amsmath}
\usepackage{graphicx}
\usepackage{fancyhdr}
\usepackage{longtable}
%\usepackage{makecell}  % $ sudo tlmgr install makecell
%\pagestyle{fancy}
%\fancyhf{}
\renewcommand{\headrulewidth}{0pt}
\setlength{\headheight}{20pt} 

\begin{document}

\title{FAIR Metric FM-I1}

\author{Mark D. Wilkinson, Susanna-Assunta Sansone, \\Erik Schultes, Peter Doorn,\\ 
Luiz Olavo Bonino da Silva Santos, Michel Dumontier}

\maketitle

\newpage





\begin{longtable}{|p{5cm}|p{9cm}|}


\hline
\emph{FIELD} & \emph{DESCRIPTION} \\
\hline
Metric Identifier &   FM-I1: \verb"https://purl.org/fair-metrics/FM_I1"
\\


\hline
Metric Name &   

Use a Knowledge Representation Language


 \\



\hline
To which principle does it apply? &   



I1 - (meta)data use a formal, accessible, shared, and broadly applicable language for knowledge representation

\\



\hline
What is being measured? & 



use of a formal, accessible, shared, and broadly applicable language for knowledge representation.


\\



\hline
Why should we measure it? & 




The unambiguous communication of knowledge and meaning (what symbols are, and how they relate to one another) necessitates the use of languages that are capable of representing these concepts in a machine-readable manner.  
  
\\



\hline
What must be provided? &  

URL to the specification of the language


\\



\hline
How do we measure it? &  

- The language must have a BNF (or other specification language) \newline 
- The URL resolves (accessible) \newline 
- The document has an IANA media-type (i.e. it is sufficiently widely-accepted and shared that it has been registered) \newline 
- The language can be arbitrarily extended (e.g. PDBml can be used to represent knowledge, but only about proteins) \newline 



\\



\hline
What is a valid result? &  


BNF (or other?) found, Media-type of the document is registered in FAIRSharing. 

Future:  FAIRSharing has tags to indicate constrained vs. extendable languages?



\\



\hline
For which digital resource(s) is this relevant? &  All\\



\hline
Examples of their application across types of digital resource &  None

\\



\hline

Comments & 

michel: there must be a syntax and associated semantics for that language.  This is sufficient \newline 
mark: there needs to be some identity or denotation in the language; (‘vanilla’) xml and json are not FAIR, so should fail this test\newline 
\newline 
*** can you (i) identify elements and (ii) make statements about them, and iii) is there a formally defined interpretation for that 
 -> HTML fails; PDF fails
\newline 
shared\newline 
-> that there are many users of the language\newline 
. acknowledged within your community\newline 
 -> hard to prove.\newline 
. could we use google to query for your filetype (can’t discriminate between different models)\newline 
-> has a media type\newline 

--> This SHOULD be stated as a IANA code [IANA-MT]\newline 


standardization of at least this listing process is a good measure of “sharedness”\newline 

broadly applicable\newline 
. that the language is extensible to a domain of interest\newline 
. you can define your own elements in accordance with the semantics of the language\newline 
\newline 
gff3 is not in the IANA list -> what steps would the community need to execute to be listed here?
cases like GFF, PDB are not broadly applicable \newline 
biopax -> is defined vnd.biopax.rdf+xml and built on rdf -> allows users to create new elements and relate them \newline 
jpg -> widely used, registered, but primarily for image content\newline 
pdf -> registered, enables users to create their own dictionary.\newline 
 

\\
\hline

\end{longtable}



\end{document}