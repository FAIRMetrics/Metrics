\documentclass[english]{article}
\usepackage[utf8]{inputenc}
\usepackage[T1]{fontenc}
\usepackage{babel}
\usepackage{color}
%\usepackage{amsmath}
\usepackage{graphicx}
\usepackage{fancyhdr}
\usepackage{longtable}
%\usepackage{makecell}  % $ sudo tlmgr install makecell
%\pagestyle{fancy}
%\fancyhf{}
\renewcommand{\headrulewidth}{0pt}
\setlength{\headheight}{20pt} 

\begin{document}

\title{FAIR Metric FM-F3}

\author{Mark D. Wilkinson, Susanna-Assunta Sansone, \\Erik Schultes, Peter Doorn,\\ 
Luiz Olavo Bonino da Silva Santos, Michel Dumontier}

\maketitle

\newpage





\begin{longtable}{|p{5cm}|p{9cm}|}


\hline
\emph{FIELD} & \emph{DESCRIPTION} \\
\hline
Metric Identifier &   FM-F3
 \\


\hline
Metric Name &   

Resource Identifier in Metadata


 \\



\hline
To which principle does it apply? &   
F3 - metadata clearly and explicitly include the identifier of the data it describes
\\



\hline
What is being measured? & 


Whether the metadata document contains the globally unique and persistent identifier for the digital resource.

\\



\hline
Why should we measure it? & 



The discovery of digital object should be possible from its metadata. For this to happen, the metadata must explicitly contain the identifier for the digital resource it describes. 
A metadata document should also not result in ambiguity about the digital object it is describing. This can be assured if the metadata document explicitly refers to the digital object by its GUID.

  
\\



\hline
What must be provided? &  


The URL of the metadata and the GUID of the digital resource it describes.


 \\



\hline
How do we measure it? &  

Parsing the metadata for the given digital resource GUID.

\\



\hline
What is a valid result? &  


Present or absent


\\



\hline
For which digital resource(s) is this relevant? &  All\\



\hline
Examples of their application across types of digital resource &  

Metadata standards (model/format/terminology)\newline 
- SBML: https://fairsharing.org/bsg-s000052 \newline 
- Dublin Core: https://fairsharing.org/bsg-s000589 \newline 
\newline 
Repositor\newline  and database
- DANS EASY (Peter) - At DANS the PID is part of the metadata, and you can only reach the actual data from the metadata. There is no reference from the PID to the actual data. \newline 
- GBIF: https://fairsharing.org/collection/GBIF \newline 
\newline 
Policies \newline 
- Wellcome Trust: https://fairsharing.org/bsg-p000019 \newline 
- Scientific Data journal: https://fairsharing.org/bsg-p000034 \newline 
\newline 
API (Michel)\newline 
- smartAPI registry API \newline 
- facebook API\newline 
\newline 
Scholarly Materials (Erik)\newline 
- Scientific Data FAIR Article (doi:10.1038/sdata.2016.18) \newline 
- Scientific Data Journal (https://www.nature.com/sdata/) \newline 
\newline 
Non-article Published Work (MarkW)\newline 
- my Zenodo Deposit for polyA (http://doi.org/10.5281/zenodo.47641)\newline 
- myExperiment Workflow (http://www.myexperiment.org/workflows/2999.html)\newline 
- Jupyter notebook on GitHub (https://github.com/\newline 
VidhyasreeRamu/GlobalClimateChange/blob/master/\newline 
GlobalWarmingAnalysis.ipynb)


\\



\hline

Comments & 

A first version of this metric would focus on just checking a URL that resolves to a document. We can’t verify that document. \newline
A second version would indicate how to structure the data policy document with a particular section (similar to how the CC licenses now have a formal structure in RDF).\newline
A third version would insist that that document and section is signed by an approved organization and made available in an appropriate repository. \\ 
\hline

\end{longtable}



\end{document}